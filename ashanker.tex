%%%%%%%%%%%%%%%%%
% This is an example CV created using altacv.cls (v1.1.3, 30 April 2017) written by
% LianTze Lim (liantze@gmail.com), based on the
% Cv created by BusinessInsider at http://www.businessinsider.my/a-sample-resume-for-marissa-mayer-2016-7/?r=US&IR=T
%
%% It may be distributed and/or modified under the
%% conditions of the LaTeX Project Public License, either version 1.3
%% of this license or (at your option) any later version.
%% The latest version of this license is in
%%    http://www.latex-project.org/lppl.txt
%% and version 1.3 or later is part of all distributions of LaTeX
%% version 2003/12/01 or later.
%%%%%%%%%%%%%%%%

%% If you want to use \orcid or the
%% academicons icons, add "academicons"
%% to the \documentclass options.
%% Then compile with XeLaTeX or LuaLaTeX.
% \documentclass[10pt,a4paper,academicons]{altacv}

%% Use the "normalphoto" option if you want a normal photo instead of cropped to a circle
% \documentclass[10pt,a4paper,normalphoto]{altacv}

\documentclass[10pt,a4paper]{altacv}

%% AltaCV uses the fontawesome and academicon fonts
%% and packages.
%% See texdoc.net/pkg/fontawecome and http://texdoc.net/pkg/academicons for full list of symbols.
%% When using the "academicons" option,
%% Compile with LuaLaTeX for best results. If you
%% want to use XeLaTeX, you may need to install
%% Academicons.ttf in your operating system's font %% folder.


% Change the page layout if you need to
\geometry{left=1cm,right=9cm,marginparwidth=6.8cm,marginparsep=1.2cm,top=1cm,bottom=1cm}

% Change the font if you want to.

% If using pdflatex:
\usepackage[utf8]{inputenc}
\usepackage[T1]{fontenc}
\usepackage[default]{lato}

% If using xelatex or lualatex:
% \setmainfont{Lato}

% Change the colours if you want to
% \definecolor{VividPurple}{HTML}{3E0097}
% \definecolor{VividPurple}{HTML}{0047AB}
\definecolor{VividPurple}{HTML}{2E4053}
\definecolor{SlateGrey}{HTML}{2E2E2E}
\definecolor{LightGrey}{HTML}{666666}
\colorlet{heading}{VividPurple}
\colorlet{accent}{VividPurple}
\colorlet{emphasis}{SlateGrey}
\colorlet{body}{LightGrey}

% Change the bullets for itemize and rating marker
% for \cvskill if you want to
\renewcommand{\itemmarker}{{\small\textbullet}}
\renewcommand{\ratingmarker}{\faCircle}

%% sample.bib contains your publications
% \addbibresource{sample.bib}

\begin{document}
\name{Apaar Shanker}
\tagline{College of Computing, Georgia Tech}
% Cropped to square from https://en.wikipedia.org/wiki/Marissa_Mayer#/media/File:Marissa_Mayer_May_2014_(cropped).jpg, CC-BY 2.0
% \photo{2.5cm}{georgia-tech-logo.jpg}
\personalinfo{%
  % Not all of these are required!
  % You can add your own with \printinfo{symbol}{detail}
  \email{ashanker9@gatech.edu}
  \phone{404-955-2251}
  % \mailaddress{935 Marietta Street, Apt. 336, 30318 Atlanta}
  \location{Atlanta, GA}
  % \homepage{marissamayr.tumblr.com/}
  % \twitter{@marissamayer}
  \linkedin{https://www.linkedin.com/in/apaar-shanker-47098252}
  \github{https://github.com/materialsinnovation} % I'm just making this up though.
% If you want to use this field (and also other academicons symbols), add "academicons" option to \documentclass{altacv}
}

%% Make the header extend all the way to the right, if you want.
\begin{fullwidth}
\makecvheader
\end{fullwidth}

%% Provide the file name containing the sidebar contents as an optional parameter to \cvsection.
%% You can always just use \marginpar{...} if you do
%% not need to align the top of the contents to any
%% \cvsection title in the "main" bar.
\cvsection[ashanker-p1sidebar]{Research Statement}

\begin{quote}
Application of Machine Learning and Big Data
techniques to
develop predictive
process-sturcture-property linkages
for automation
and accelaration of material manufacturing
cycle.
\end{quote}

\cvsection{Projects}

\cvevent{PyMKS Python Project Development}{NIST, Georgia Institute of Technology}{May 2017 -- Ongoing}{Gaithersburg, MD}
\begin{itemize}
\item \github{https://github.com/wd15/fmks}
\item Implemented the Material Knowledge Systems algorithms in functional python using Pytoolz and Dask
to leverage multiprocessing and multithreading capabilities of modern PCs.
\end{itemize}

\divider

\cvevent{High Throughput Selection of 2D Nanoporous Zeolites}{NSF, Georgia Institute of Technology}{Sep 2016 -- ongoing}{Georgia Tech, Atlanta}
\begin{itemize}
\item Working on the development of atomistic structure descriptors for nano-porous
materials.
\item Working on the development of structure-property linkages for high throughput selection of
nanoporous zeolites for desired catalytic or separation attributes.
\end{itemize}

\divider

\cvevent{Surrogate, Predictive models for Microstructure Evolution}{NIST, Georgia Institute of Technology}{Jan 2017 -- ongoing}{Georgia Tech, Atlanta}
\begin{itemize}
\item Working on the development of process-structure linkages based on phasefield microstructure evolution models.
\end{itemize}
\divider

\cvevent{Modeling Alloy Solidification in presence of Convection}{Indian Institute of Science}{Sep 2014 -- May 2016}{Bangalore, India}
\begin{itemize}
\item Developed a multiphysics Phasefield solver over 15 months to simulate alloy solidification in presence of fluid flow.
\item Coded entirely in C,with full
parallelization implemented using \textbf{MPI}
and run on around 1000 processors at the institute supercomputing facility.
\end{itemize}
\divider

\cvevent{Gas Turbine Blades Repair}{GE India Technology Center}{May 2014 -- Sep 2014}{Bangalore, India}

\begin{itemize}
\item Modified repair protocols for the industrial Gas Turbine frames based on structural and material analysis of components resulting in savings to the tune of \$2 million for the company.
\end{itemize}

% \cvsection{Core Research Areas}
%
% \wheelchart{1.5cm}{0.5cm}{%
% 10/13em/accent!30/Fourier Analysis and Image Processing,
% 25/9em/accent!60/Spatial Statistics,
% 5/12em/accent!10/Materials Engineering,
% 20/12em/accent!40/Thermal and Solid State Physics,
% 30/9em/accent/Machine Learning,
% 5/8em/accent!20/Partial Differential Equation numerical solvers
% }

\clearpage
\end{document}
